\documentclass{article}
\usepackage[utf8]{inputenc}

\title{DATA.SCIENCE.PROJECT-PS11_Quick}
\author{Sam Quick}
\date{April 2019}

\usepackage{natbib}
\usepackage{graphicx}

\begin{document}
\maketitle

\section{Introduction}
The purpose of my research is to assess the relationships that exist, if any, between energy consumption and economic growth. As many countries develop their economies globally we are able to study the characteristics that may accompany said growth. In doing so, we are able to glean lessons for improving the growth process for other countries and our own. Historically, technological advances that allowed us to harness more dense and efficient energy sources have led to the largest percentage growth periods in world history. Policy makers can use research as this inform their later decisions related to energy, which is very prudent considering the Oklahoma economy is so dependent on its energy sector.

\citep{asafu2000relationship}
\citep{zhang2009energy}
\citep{paul2004causality}
\citep{mehrara2007energy}
\citep{ritchie1981complexities}


\section{Literature Review}
There are a lot of research papers that have studied the correlation and relationships between energy topics and economic growth. This section of the research will summarize some of the most relevant papers and provide a summary that will relate said topics and research to this paper's research.
 
\section{Data}
The data is pulled from the Federal Reserve and the Energy Information Administration thus far. I am attempting to find *Reliable* data on China, India, and certain African Countries that will allow me to look at countries at varying levels of economic growth stages.

\section{Empirical Methods}
This section will be geared towards analyzing the relationship between net national growth of three countries with their respective levels of energy consumption. This research will assess that relationship in a few different  ways. Firstly, it will look at the relationship between the two variables over time. Secondly, it will assess the significance of the relationship   over a time series analysis. Finally, I will decompose certain variables that will allow me to see the main variables and sectors that drive growth the most.

\section{Research Findings}
Here I will present any findings that I have, both supportive and non-supportive. 

\section{Conclusion}
The results and the  conclusions of the data will be summarized here. 

\bibliographystyle{plain}
\bibliography{Quick_References}

\end{document}
